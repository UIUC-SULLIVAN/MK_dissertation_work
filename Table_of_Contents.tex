\documentclass[thesis,tocnosub,noragright,centerchapter,12pt,fullpage]{uiucecethesis09}

% Use draftthesis for notes and date markings on every page.  Useful when you
%   have multiple copies floating around.
% Use offcenter for the extra .5 inch on the left side. Needed with fullpage and fancy.
% Use mixcasechap for compatibility with hyperref package, which does NOT like all caps default
% Use edeposit for the adviser/committee on the title page.
% Use tocnosub to suppress subsection and lower entries in the TOC.
% PhD candidates use "proquest" for the proquest abstract.

\makeatletter

%%%%%%%%%%%%%%%
% MK Added to make bib work
% \bstctlcite{IEEEexample:BSTcontrol}

% MK added to highlight stuff
\usepackage{color,soul}

% MK added so figures can use the [H] option and be put where I want them.
\usepackage{float}

% MK added so algorithms work
\usepackage{algorithm}

%%%%%%%%%%%%%%%

\setcounter{secnumdepth}{5} % to make subsubsections work
\usepackage{setspace}
\usepackage{epsfig}  % for figures
\usepackage{graphicx}  % another package that works for figures
\usepackage{subfigure}  % for subfigures
\usepackage{amsmath}  % for math spacing
\usepackage{amssymb}  % for math spacing
%\usepackage{url}  % Hyphenation of URLs.
\usepackage{lscape}  % Useful for wide tables or figures.
\usepackage[justification=raggedright]{caption}	% makes captions ragged right - thanks to Bryce Lobdell


\DeclareMathOperator*{\argminA}{arg\,min} % Jan Hlavacek

% Uncomment the appropriate one of the following four lines:
\phdthesis
%\phdthesis
%\otherdoctorate[abbrev]{Title of Degree}
%\othermasters[abbrev]{Title of Degree}




\begin{document}

%%%%%%%%%%%%%%%%%%%%%%%%%%%%%%%%%%%%%%%%%%%%%%%%%%%%%%%%%%%%%%%%%%%%%%%%%%%%%%%
% COPYRIGHT
%
%\copyrightpage
%\blankpage

%%%%%%%%%%%%%%%%%%%%%%%%%%%%%%%%%%%%%%%%%%%%%%%%%%%%%%%%%%%%%%%%%%%%%%%%%%%%%%%
% TITLE
%


%\


\tableofcontents



\chapter{Introduction}











\chapter{Literature Review}


\section{Scintillation Detector Physics}




\section{Isotope Identification}




\section{Isotope Identification Using ANNs}




\section{Identification Tasks}

\subsection{Source Search in a City}


\subsection{Plutonium and Uranium Enrichment Measurements}




















\subsection{Post-Detonation Debris Analysis}


\chapter{Previous Work}

\section{Current ANN Method}


\section{Non-shielded ANN Results}

\subsection{Thesis Results}

\subsection{Published Work (Thesis extension)}


\section{Shielded ANN Results}


\section{Uranium ANN with Autoencoders Result}

\subsection{CVT 2017 Results}




\chapter{Future Work and Proposed Experiments}


\section{Introduction}


\section{Find appropriate Hyperparameter Bounds to Prove the 60 Samples Argument}

\section{On the Usefulness of Autoencoders and PCA}

\subsection{Non-shielded Autoencoder with 32 isotope library - Simple Method and Calibration Drift-Perfect Spectrum}

\subsection{Shielded Autoencoder - Simple Method and Shield-Perfect Spectrum}

\subsection{U/Pu Autoencoder - Simple Method and Shield-Perfect Spectrum}

\subsection{Post-Det Surrogate Autoencoder - Simple Method and Calibration Drift-Perfect Spectrum}



\end{document}

